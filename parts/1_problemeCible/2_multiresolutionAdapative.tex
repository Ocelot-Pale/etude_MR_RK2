La multirésolution adaptative consiste à compresser le maillage, puis a effectuer les calculs sur le maillage compressé.
Le schéma classique est le suivant : 
\begin{enumerate}
    \item Partir d'un état compressé au pas de temps $n$.
    \item Calculer la solution au pas de temps $n+1$
    \item Compresser de nouveau selon un seuil de compression $\varepsilon$ grâce à une transofrmée multiéchelle.
\end{enumerate}
Lors de la compression, la transofrmée multiéchelle représente la solution sur plusieurs niveaux de détails, du plus global, au plus local.
Plus le niveau est profond, c'est à dire plus il est local, moins les détails associés portent d'information. 
L'opération de compression est réalisée en suprimant en chaque cellules, les niveaux dont la valeur des détails
passent sous un certain seuil \footnote{Typiquement $2^{\Delta l} \varepsilon$ où ${\Delta l}$} \cite{postelApprox} .

Ce seuil $\varepsilon$ n'est pas l'unique juge lors la compression, des heuristiques reposant sur la quantité d'information des détails de niveau supérieur
sont utilisées pour ne pas seuiller systématiquement. L'objectif est en quelque sorte d'anticiper le besoin de détails\footnote{Même si la quantité d'information laisse entendre que 
certains détails pourraient être ignorés, l'intuition phyisque pose sont véto et force certains détails à être conservés par précaution, par exemple si un front front d'onde arrive.}.
La plus connue est l'heuristique d'Ami Harten \cite{harten1994}.

Plusieurs stratégies existent pour réaliser le calcul d'un pas de temps à l'autre. Généralement, on estime les quantités au temps $n+1$ aux niveaux courrants, 
à partir des quantité au niveau courrant au temps $n$. Ensuite une opération de reconstruction-prédiction détermine le niveau de finesse requis de la solution au temps $n+1$.
Il est également possible de calculer les quantités du temps $n+1$ au niveau courrant, à partir 
des quantités au temps $n$ \textbf{reconstruites à un niveau plus fin}. Bien que cela aie une faible efficacité computationelle, cela reduirait les erreurs liés à la multirésolution
selon la qualité du prédicteur employé comme discuté en \cite{belloti_et_al_2025}. Ici nous allons étudier théoriquement les erreurs dans un contexte similaire. 
Nous nous plaçons sur une cellule à un niveau de détail fixé, les flux sont calculés à partir de quantités reconstruites à un niveau de détails $\Delta l$ plus fin.
Le raisonnement et les ressources de caclul formel de \cite{belloti_et_al_2025} on été d'une aide précieuse.

\paragraph{Calcul du flux au travers de $\Delta l$ niveaux}
Lorsque l'on applique le procédé de mulutirésolution, étant donné une cellule à un niveau de détail donné $l$, on cherche à faire évoluter la valeur à l'étape $n$ vers la valeur à l'étape $n+1$. 
Pour ce faire, il faut évaluer les flux à partir les cellules voisines. Dès lors plusieurs choix s'offrent à nous. Où bien on utilise les cellules voisines à leurs niveaux courrants, où bien on use de l'opérateur 
de reconstruction afin d'estimer les cellules voisines à des niveaux plus fins.\par
Dans un premier temps le stecnil est choisi égal à 1. L'opérateur de prédiction d'un niveau à l'autre s'écrit alors : 
\begin{align}
    \hat u^{l+1}_{2k} &= +\frac{1}{8} u^l_{k-1} + u^l_k - \frac{1}{8} u^l_{k+1},\\
    \hat u^{l+1}_{2k+1} &= -\frac{1}{8} u^l_{k-1} + u^l_k + \frac{1}{8} u^l_{k+1}.
\end{align}
Puis en notant $\doublehat{u}^{l+\Delta l}_{(\cdot)}$ cet opérateur de prédiction itéré au travers de $\Delta l$ niveaux\footnote{
    Au sens où l'on applique le prédicteur à des données déjà issues d'une prédiction.
} : 
\begin{align}
    \begin{bmatrix}
        \doublehat{u}^{(l+\Delta l)}_{2^{\Delta l}k-2}\\
        \doublehat{u}^{(l+\Delta l)}_{2^{\Delta l}k-1}\\
        \doublehat{u}^{(l+\Delta l)}_{2^{\Delta l}k}\\
        \doublehat{u}^{(l+\Delta l)}_{2^{\Delta l}k+1}\\
    \end{bmatrix}
        =
    \underbrace{
    \begin{bmatrix}
        +1/8 & 1 & -1/8 & 0 \\
        -1/8 & 1 & +1/8 & 0 \\
        0 & +1/8 & 1 & -1/8 \\
        0 & -1/8 & 1 & +1/8 
    \end{bmatrix}^{\Delta l}}_{\text{Matrice de passage } P \text{ pour }s=1.}
    \cdot
    \begin{bmatrix}
        u^l_{k-2}\\
        u^l_{k-1}\\
        u^l_{k}\\
        u^l_{k+1}\\
    \end{bmatrix}
\end{align}

En particulier, si la cellule étudiée est au niveau courrant $l$ alors on choisira d'aller approximer le flux au niveau le plus fin, c'est à dire avec $\dlbar = \bar l - l$.
Dès lors les flux approximés au niveau fins sont : 
\begin{align}
    \doublehat{\Phi}_{k-1/2} &=  \doublehat{u}^{l+\dlbar}_{2^\dlbar k} -  \doublehat{u}^{l+\dlbar}_{2^\dlbar k-1} + \frac{1}{2} \lambda \Bigl(
         \doublehat{u}^{l+\dlbar}_{2^\dlbar k+1}
         - 3 \doublehat{u}^{l+\dlbar}_{2^\dlbar k}
         + 3 \doublehat{u}^{l+\dlbar}_{2^\dlbar k-1}
         - \doublehat{u}^{l+\dlbar}_{2^\dlbar k-2}
    \Bigr),\\
    \doublehat{\Phi}_{k+1/2} &=  \doublehat{u}^{l+\dlbar}_{2^\dlbar (k+1)} -  \doublehat{u}^{l+\dlbar}_{2^\dlbar (k+1)-1} + \frac{1}{2} \lambda \Bigl(
         \doublehat{u}^{l+\dlbar}_{2^\dlbar (k+1)+1}
         - 3 \doublehat{u}^{l+\dlbar}_{2^\dlbar (k+1)}
         + 3 \doublehat{u}^{l+\dlbar}_{2^\dlbar (k+1)-1}
         - \doublehat{u}^{l+\dlbar}_{2^\dlbar (k+1)-2}
    \Bigr)
\end{align}
Cela s'écrit sous la forme matricielle suivante (utile pour utiliser les outils de calcul formel).
\begin{align}
    \doublehat{\Phi}_{k-1/2}
        &=
    \begin{bmatrix}
        -\frac{\lambda}{2}&
        (\frac{3}{2} \lambda - 1)&
        (1 - \frac{3}{2} \lambda)&
        \frac{\lambda}{2}&
    \end{bmatrix}
    \begin{bmatrix}
        +1/8 & 1 & -1/8 & 0\\
        -1/8 & 1 & +1/8 & 0\\
        0 & +1/8 & 1 & -1/8\\
        0 & -1/8 & 1 & +1/8\\
    \end{bmatrix}^{\dlbar}
    \begin{bmatrix}
        u^l_{k-2}\\
        u^l_{k-1}\\
        u^l_{k}\\
        u^l_{k+1}\\
    \end{bmatrix}
\end{align}
\begin{align}
    \doublehat{\Phi}_{k+1/2}
        &=
    \begin{bmatrix}
        -\frac{\lambda}{2}&
        (\frac{3}{2} \lambda - 1)&
        (1 - \frac{3}{2} \lambda)&
        \frac{\lambda}{2}&
    \end{bmatrix}
    \begin{bmatrix}
        +1/8 & 1 & -1/8 & 0\\
        -1/8 & 1 & +1/8 & 0\\
        0 & +1/8 & 1 & -1/8\\
        0 & -1/8 & 1 & +1/8\\
    \end{bmatrix}^{\dlbar}
    \begin{bmatrix}
        u^l_{k-1}\\
        u^l_{k}\\
        u^l_{k+1}\\
        u^l_{k+2}\\
    \end{bmatrix}.
\end{align}

Attention le schéma final est légèrement différent car il fait ici intervenir deux pas d'espace: $\Delta x$ le pas au niveau le plus fin
et $\Tilde {\Delta x} = 2^{\Delta l} \Delta x$ le pas du niveau courrant. Ainsi le schéma final est :
\begin{align}
    {u}^{n+1}_k = {u}^n_k + \frac{\lambda}{2^{\Delta l}} \Bigl( \doublehat{\Phi}^n_{k+1/2} - \doublehat{\Phi}^n_{k-1/2} \Bigr)
\end{align}