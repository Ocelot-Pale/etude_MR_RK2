Pour résoudre cette équation aux dérivées partielles, nous utilisons une méthode des lignes. 
D'abord un schéma volume fini pour la discrétisation spatiale menant à l'équation semi-discrétisée suivante : 
\begin{align}
    \dt{U}(t) = \frac{D}{\Delta x} \Bigl(\frac{U_{k+1} - 2 U_{k} + U_{k-1}}{\Delta x}\Bigr)
\end{align}
Puis une méthode de Runge und Kutta explicite d'ordre deux sur l'opératur linéaire donne :
\begin{align}
    U_k^{n+1} &= U_k^n\\ \notag
    &+ D \frac{\Delta t}{\Delta x} \Bigl(\frac{U_{k+1} - 2 U_{k} + U_{k-1}}{\Delta x}\Bigr)\\ \notag
    &+ D^2 \frac{\Delta t^2}{\Delta x^2} \Bigl(\frac{U_{k+2} -4 U_{k+1}  +6 U_{k} -4 U_{k-1} + U_{k-2}}{\Delta x^2}\Bigr).
\end{align}
Cela se remet sous la forme d'un schéma conservatif sous du type:
\begin{align}
    u^{n+1}_k = u^n_k + \lambda \Bigl( \Phi^n_{k+1/2} - \Phi^n_{k-1/2} \Bigr)
\end{align}
Avec : 
\begin{align}
    \lambda = D \frac{\Delta t}{\Delta x^2}
\end{align}

\begin{align}
    \Phi^n_{k+1/2} = u^n_{k+1} - u^n_{k} + \frac{1}{2} \lambda \bigr( u^n_{k+2} - 3  u^n_{k+1} + 3 u^n_{k} - u^n_{k-1} \bigl),\\
    \Phi^n_{k-1/2} = u^n_{k} - u^n_{k-1} + \frac{1}{2} \lambda \bigr( u^n_{k+1} - 3  u^n_{k} + 3 u^n_{k-1} - u^n_{k-2} \bigl).  
\end{align}

