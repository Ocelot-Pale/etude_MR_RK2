\paragraph{Première étape : développement de Taylor}
Dans un premier temps, on suppose l'existence d'une fonction assez régulière vérifiant le schéma.
Dans le schéma numérique les termes $u^{n+\delta_t}_{k+\delta_x}$ sont remplacés par leur pendant
continus: $u(x+\delta_x \Delta x, t + \delta_t \Delta t)$. Puis on réalise le dévelopmment de Taylor suivant : 
\begin{align}
    u(x+\delta_x \Delta x, t + \delta_t \Delta t) &= u(x+\delta_x \Delta x , t) 
        + \sum_{i} \frac{(\delta_x \Delta x )^j}{j!} \frac{\partial^j u}{\partial t^j }(x+\delta_x \Delta x)\\\notag
    &= u(x , t) 
        + \sum_i  \frac{(\delta_x \Delta x)^i}{i!} 
                \frac{\partial^i u}{\partial x^i}u(x,t)
        + \sum_j  \frac{(\delta_x \Delta x )^j}{j!} 
                \frac{\partial^j u}{\partial^j t}(x,t)
        + \sum_{i,j}  \frac{(\delta_x \Delta x )^{i}(\delta_t \Delta t )^{j}}  {i! \times j!}
                \frac{\partial^{j+i} u}     {\partial t^j \partial x^i}(x,t)
\end{align}
L'équation au dérivée partielles qui apparait est l'équation équivalente. libre à nous de la tronquer à l'ordre qui nous convient.
\paragraph{Deuxième étape : procédure de Cauchy-Kovaleskaya (optionnel)}
Afin d'enrichir l'analyse, et permettre le développement de schéma couplé espace-temps, l'étape précédente est généralement suivie d'une procédure de Cauchy-Kovaleskaya.
La procédure de Cauchy-Kovaleskaya consiste à utiliser la rélation entre les dérivées en espace et en temps données par l'équation cible et remplacer les dérivées en temps par des dérivées en espace dans l'équation équivalente. 
Cela consiste simple à partir de l'équation : $\dt{u} = D \dxx{u}$ et décrire de manière itérée : 
\begin{align}
    \dtt{u} &= D^2 \frac{\partial^4 u }{\partial x^4}\\\notag
    \dttt{u}&= D^3 \frac{\partial^6 u }{\partial x^6}\\\notag
    &\vdots\\\notag
    \frac{\partial^n u}{\partial x^n} &=D^n \frac{\partial^{2n} u}{\partial x^{2n}}
\end{align}
\paragraph{Ultime étape étape : relation entre le pas de temps et d'espace (optionnel)}
Lorsque l'on utilise un schéma en pratique, il est courrant de supposer une relation entre les pas d'espace et de temps, par exmeple une condition de stabilité du type 
$\Delta t \propto \frac{\Delta x^2}{D}$.
Il est donc très utile d'injecter cette relation dans l'équation équivalente pour comprendre le comportement du schéma en conditions réelles.