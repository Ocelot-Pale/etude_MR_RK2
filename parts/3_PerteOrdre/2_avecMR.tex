L'équation modifiée avec multirésolution, avant procédure de Cauchy-Kovaleskaya est:
\begin{align}
    \frac{\partial u}{\partial  t} =&D \frac{\partial^{2} u}{\partial x^{2}}\\\notag
    &+ \frac{\Delta t}{2} \underbrace{\Bigl(2^{2 \Delta l} D^{2}           \frac{\partial^{4} u}{\partial x^{4}} -\frac{\partial^{2} u}{\partial t^{2}} \Bigr)}_{\text{Ne se compensent plus.}}
    -\frac{\Delta t^{3}}{24}                          \frac{\partial^{4} u}{\partial t^{4}} 
    - \frac{\Delta t^{2}}{6}                           \frac{\partial^{3} u}{\partial t^{3}}
    +\frac{\Delta x^{2}}{12} (1 - 3\Delta l)    2^{2 \Delta l} D \frac{\partial^{4} u}{\partial x^{4}}
\end{align}
Dans ce cas le terme en facteur du $\Delta t$ ne se s'annule plus. En effet le terme $D^{2}\frac{\partial^{4} u}{\partial x^{4}}$ est devenu au cours de la reconstruction
$2^{2 \Delta l} D^{2}\frac{\partial^{4} u}{\partial x^{4}}$. En conséquence, la méthode perds un ordre de convergence temporel.\par
Ce mécanisme s'explique de la manière suivante: dans l'équation équivalente, le terme $\frac{\partial^{2} u}{\partial t^{2}}$ apparaît indépendemment de la discrétisation spatiale
\footnote{Il emerge de la différence $u_k^{n+1} - u_k^{n}$ à $k$ fixé.}. La méthode des lignes initiale crée un terme \textit{sur mesure} pour le compenser en approximant le terme spatial
$D^{2}\frac{\partial^{4} u}{\partial x^{4}}$. Cepednant au cours du processus de compression-reconstruction, cette apprximation est entachée d'un facteur $2^{2 \Delta l}$.
En d'autres termes le terme spatial construit pour compenser un terme temporel a été modifié par la multi-résolution, alors que le terme temporel lui n'est pas affecté par la multirésolution.
Ainsi, les deux termes ne se compensent plus et l'ordre est perdu.