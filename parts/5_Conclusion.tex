Cette étude mets en valeur qu'une méthode des lignes très simple, dans un contexte de multirésolution-adaptative
petu mener à des couplages des erreurs espace-temps polluant l'ordre initial de la méthode.
En particulier la reconstruction altère des termes spatiaux qui ne compensent plus certaines erreurs temporelles. 
Pour l'heure il conviendrait d'étudier ce phénomène sur d'autre méthodes, d'autres équations et surtout de l'étudier expérimentalement.
Je ne parviens pas à mettre en lumière ce phénomène expérimentalement. 
D'abord ce n'est pas possible de reproduire fidèlement cette situation car en pratique la condition de stabilité force $\Delta t \propto \Delta x^2$. 
Ensuite j'ai essayé de mettre en lumière le phénomène sur une méthode implicite (RKI2 SDIRK) mais le phénomène n'a pas été observé. 
...     à suivre    ... 