\documentclass[11pt]{article}

% Encodage & langue
\usepackage[utf8]{inputenc}
\usepackage[T1]{fontenc}
\usepackage[french]{babel}

\usepackage{tikz}
\usepackage{tikz-cd}
\usetikzlibrary{arrows.meta, positioning, calc}

% Maths
\usepackage{amssymb, amsthm}
\usepackage{amsmath}
\usepackage[backend=biber, style=numeric, citestyle=numeric, maxnames=3]{biblatex}
\addbibresource{references.bib}
% Marges
\usepackage[a4paper,margin=1.5cm]{geometry}

% Liens
\usepackage[colorlinks=true, linkcolor=blue, citecolor=blue, urlcolor=blue]{hyperref}

% Figures
\usepackage{graphicx}
\usepackage{float}

% Théorèmes
\newtheorem{theoreme}{Théorème}[section]
\newtheorem{proposition}[theoreme]{Proposition}
\newtheorem{lemme}[theoreme]{Lemme}
\newtheorem{corollaire}[theoreme]{Corollaire}
\theoremstyle{definition}
\newtheorem{definition}[theoreme]{Définition}
\newtheorem{exemple}[theoreme]{Exemple}
\theoremstyle{remark}
\newtheorem{remarque}[theoreme]{Remarque}

% Environnement pour les preuves
\newcommand{\pt}{\partial_t}
\newcommand{\px}{\partial_x}
\newcommand{\pxx}{\partial_x^{(2)}}
\newcommand{\pxxx}{\partial_x^{(3)}}
\newcommand{\pxxxx}{\partial_x^{(4)}}
\newcommand{\dgot}{ \text{\gothfamily{D}}}
\newcommand{\doublehat}[1]{\widehat{\widehat{#1}}}
\newcommand{\dlbar}{{\overline{\Delta l}}}

% Titres
\title{Impact théorique de la multirésolution adaptative sur la convergence temporelle les méthodes des lignes}
\author{Alexandre Edeline}%\\ \small{ENSTA Paris, adresse email}}
\date{\today}
% Dérivées temporelles
\newcommand{\dt}[1]{\partial_t #1}
\newcommand{\dtt}[1]{\partial_{tt} #1}
\newcommand{\dttt}[1]{\partial_{ttt} #1}

% Dérivées spatiales (selon x)
\newcommand{\dx}[1]{\partial_x #1}
\newcommand{\dxx}[1]{\partial_{xx} #1}
\newcommand{\dxxx}[1]{\partial_{xxx} #1}

% Variante avec dérivée totale si besoin
\newcommand{\Dt}[1]{\frac{d #1}{dt}}
\newcommand{\Dtt}[1]{\frac{d^2 #1}{dt^2}}
\newcommand{\Dx}[1]{\frac{d #1}{dx}}
\newcommand{\Dxx}[1]{\frac{d^2 #1}{dx^2}}

\begin{document}

\maketitle

\begin{abstract}
Nous analysons, sur un cas test simple, les effets de la multirésolution adaptative sur l'ordre de convergence temporel d'une méthode des lignes. Nous montrons que les erreurs induites par la structure adaptative peuvent polluer l'intégration en temps, réduisant l'ordre global de la méthode.
\end{abstract}

\section{Introduction}
    Nous allons résoudre un problème classique de diffusion par méthode des lignes avec multirésolution adaptative.
La multirésolution est une méthode de compression de maillage très efficace pour accéléler les calculs et réduire les contraintes mémoires.
Elle repose sur une compression par transformée en ondelette et sur des interpolateurs polynomiaux permettant une reconstruction-prédiction
les données au niveau de précision choisit\footnote{C'est l'opération permettant de décompresser l'information}.\\
La multirésolution adaptative détruit donc intrisèqement l'information en deçà d'un certain seuil\footnote{Le seuil de compression}. De même 
l'opérateur de reconstruction-prédiction n'est pas exacte, son ordre dépend d'un \textit{stencil}
\footnote{Le stencil désigne le nombre de cellules entrée du prédicteur. Un grand stencil donne un bon ordre de précision mais le coût de la reconstruction
augmente linéairement avec ce dernier.} Ainsi la multirésolution vient aux prix de deux sources d'erreur pilotées par $\varepsilon$ le seuil de compression
et par $s$ le stencil de l'opérateur de reconstruction-prédiction.\\
Nous nous proposons d'étudier, en calculant des équations équivalente les impacts théoriques de la multirésolution adaptative 
sur une méthode des lignes pour la diffusion.
Il semble que la multirésolution adaptative pourrait altérer l'ordre de convergence temporel des méthodes des lignes.
\newpage
\section{Problème cible}
    Nous cherchons à résoudre le problème de diffusion suivant :
    \begin{align}
        \dt{u} = D \dxx{u}.
    \end{align}
    Nous ignorons les problèmatiques de conditions de bords.
        \subsection{Méthode des lignes utilisée}
            Pour résoudre cette équation aux dérivées partielles, nous utilisons une méthode des lignes. 
D'abord un schéma volume fini pour la discrétisation spatiale menant à l'équation semi-discrétisée suivante : 
\begin{align}
    \dt{U}(t) = \frac{D}{\Delta x} \Bigl(\frac{U_{k+1} - 2 U_{k} + U_{k-1}}{\Delta x}\Bigr)
\end{align}
Puis une méthode de Runge und Kutta explicite d'ordre deux sur l'opératur linéaire donne :
\begin{align}
    U_k^{n+1} &= U_k^n\\ \notag
    &+ D \frac{\Delta t}{\Delta x} \Bigl(\frac{U_{k+1} - 2 U_{k} + U_{k-1}}{\Delta x}\Bigr)\\ \notag
    &+ D^2 \frac{\Delta t^2}{\Delta x^2} \Bigl(\frac{U_{k+2} -4 U_{k+1}  +6 U_{k} -4 U_{k-1} + U_{k-2}}{\Delta x^2}\Bigr).
\end{align}
Cela s'écrit sous la forme conservative suivante:
\begin{align}
    u^{n+1}_k = u^n_k + \lambda \Bigl( \Phi^n_{k+1/2} - \Phi^n_{k-1/2} \Bigr)
\end{align}
Avec : 
\begin{align}
    \lambda = D \frac{\Delta t}{\Delta x^2}
\end{align}

\begin{align}
    \Phi^n_{k+1/2} = u^n_{k+1} - u^n_{k} + \frac{1}{2} \lambda \bigr( u^n_{k+2} - 3  u^n_{k+1} + 3 u^n_{k} - u^n_{k-1} \bigl),\\
    \Phi^n_{k-1/2} = u^n_{k} - u^n_{k-1} + \frac{1}{2} \lambda \bigr( u^n_{k+1} - 3  u^n_{k} + 3 u^n_{k-1} - u^n_{k-2} \bigl).  
\end{align}


        \subsection{multirésolution adaptative}
            La multirésolution adaptative consiste à compresser le maillage, puis a effectuer les calculs sur le maillage compressé.
Le schéma classique est le suivant : 
\begin{enumerate}
    \item Partir d'un état compressé au pas de temps $n$.
    \item Calculer la solution au pas de temps $n+1$
    \item Compresser de nouveau selon un seuil de compression $\varepsilon$ grâce à une transofrmée multiéchelle.
\end{enumerate}
Lors de la compression, transofrmée multiéchelle représente la solution sur plusieurs niveaux de détails, du plus global, au plus local.
Plus le niveau est profond, c'est à dire plus il est local, moins les détails associés portent d'information. Alors, s'ils passent sous un certain seuil
\footnote{Typiquement $2^{\Delta l} \varepsilon$ où ${\Delta l}$ est le niveau de détail en question et $\varepsilon$ le seuil de compression global fixé par l'utilisateur.}
, ils sont ignorés\cite{postelApprox} c'est à dire compressés.
Ce seuil $\varepsilon$ n'est pas l'unique juge pour la compression, des heuristiques reposant sur la quantité d'information des détails de niveau supérieur
sont utilisées pour ne pas seuiller systématiquement. L'objectif est en quelque sorte d'anticiper le besoin de détails\footnote{Même si la quantité d'information laisse entendre que 
certains détails ne sont pas nécessaires, l'intuition phyisque pose sont véto et forcent certains détails à être conservés en prévention par exemple de l'arrivée d'un front d'onde.}.
La plus connue est l'heuristique d'Ami Harten \cite{harten1994}.
Plusieurs stratégies sont utilisées pour réaliser le calcul d'un pas de temps à l'autre. Généralement, on estime les quantités au temps $n+1$ aux niveaux courrants, 
à partir des quantité au niveau courrant au temps $n$. Ensuite une opération de reconstruction-prédiction détermine le niveau de finesse requis de la solution au temps $n+1$.
Il serait envisageable par exemple de calculer les quantités du temps $n+1$ au niveau courrant, à partir 
des quantités au temps $n$ \textbf{reconstruites à un niveau plus fin}. Bien que cela aie une faible efficacité computationelle, cela reduirait les erreurs liés à la multirésolution
selon la qualité du prédicteur employé comme discuté en \cite{belloti_et_al_2025}. Ici nous allons étudier théoriquement les erreurs dans un contexte similaire. 
Nous nous placons sur une cellule à un niveau de détail fixé,l'on calcule les flux à partir de quantités reconstruites à un niveau de détails $\Delta l$ plus fin.
Le raisonnement et les ressources de caclul formel de \cite{belloti_et_al_2025} on été d'une aide précieuse.

\paragraph{Calcul du flux au travers de $\Delta l$ niveaux}
Lorsque l'on applique le procédé de mulutirésolution, étant donné une cellule à un niveau de détail donné $l$, on cherche à faire évoluter la valeur à l'étape $n$ vers la valeur à l'étape $n+1$. 
Pour ce faire, il faut évaluer les flux à partir les cellules voisines. Dès lors plusieurs choix s'offrent à nous. Où bien on utilise les cellules voisines à leurs niveaux courrants, où bien on use de l'opérateur 
de reconstruction afin d'estimer les cellules voisines à des niveaux plus fins.\par
Dans un premier temps le stecnil est choisi égal à 1. L'opérateur de prédiction d'un niveau à l'autre s'écrit alors : 
\begin{align*}
    \hat u^{l+1}_{2k} &= +\frac{1}{8} u^l_{k-1} + u^l_k - \frac{1}{8} u^l_{k+1},\\
    \hat u^{l+1}_{2k+1} &= -\frac{1}{8} u^l_{k-1} + u^l_k + \frac{1}{8} u^l_{k+1}.
\end{align*}
Puis en notant $\doublehat{u}^{l+\Delta l}_{(\cdot)}$ cet opérateur de prédiction itéré au travers de $\Delta l$ niveaux\footnote{
    Au sens où l'on applique le prédicteur à des données déjà issues d'une prédiction.
} : 
\begin{align*}
    \begin{bmatrix}
        \doublehat{u}^{(l+\Delta l)}_{2^{\Delta l}k-2}\\
        \doublehat{u}^{(l+\Delta l)}_{2^{\Delta l}k-1}\\
        \doublehat{u}^{(l+\Delta l)}_{2^{\Delta l}k}\\
        \doublehat{u}^{(l+\Delta l)}_{2^{\Delta l}k+1}\\
    \end{bmatrix}
        =
    \underbrace{
    \begin{bmatrix}
        +1/8 & 1 & -1/8 & 0 \\
        -1/8 & 1 & +1/8 & 0 \\
        0 & +1/8 & 1 & -1/8 \\
        0 & -1/8 & 1 & +1/8 
    \end{bmatrix}^{\Delta l}}_{\text{Matrice de passage } P \text{ pour }s=1.}
    \cdot
    \begin{bmatrix}
        u^l_{k-2}\\
        u^l_{k-1}\\
        u^l_{k}\\
        u^l_{k+1}\\
    \end{bmatrix}
\end{align*}

En particulier, si la cellule étudiée est au niveau courrant $l$ alors on choisira d'aller approximer le flux au niveau le plus fin, c'est à dire avec $\dlbar = \bar l - l$.
Dès lors les flux approximés au niveau fins sont : 
\begin{align}
    \doublehat{\Phi}_{k-1/2} &=  u^{l+\dlbar}_{2^\dlbar k} -  u^{l+\dlbar}_{2^\dlbar k-1} + \frac{1}{2} \lambda \bigl(
         u^{l+\dlbar}_{2^\dlbar k+1}
         - 3 u^{l+\dlbar}_{2^\dlbar k}
         + 3 u^{l+\dlbar}_{2^\dlbar k-1}
         - u^{l+\dlbar}_{2^\dlbar k-2}
    \bigr),\\
    \doublehat{\Phi}_{k+1/2} &=  u^{l+\dlbar}_{2^\dlbar (k+1)} -  u^{l+\dlbar}_{2^\dlbar (k+1)-1} + \frac{1}{2} \lambda \bigl(
         u^{l+\dlbar}_{2^\dlbar (k+1)+1}
         - 3 u^{l+\dlbar}_{2^\dlbar (k+1)}
         + 3 u^{l+\dlbar}_{2^\dlbar (k+1)-1}
         - u^{l+\dlbar}_{2^\dlbar (k+1)-2}
    \bigr)
\end{align}
Attention ici $\lambda$ n'est plus exactement le nombre de Courrant mais un nombre de Courrant effectif.
En effet la différence de flux est normalisée par la taille de la cellule du niveau courrant, c'est à dire : $2^{\Delta l} \Delta x$, en revanche, le gradient d'une interface à l'autre est calculé au niveau le plus fin donc la différence entre les cellules de part et d'autre de l'interface est $\Delta x$ (niveau le plus fin). 
Ainsi dans ce qui suite: $\lambda = \frac{1}{2^{\Delta l}} \frac{D \Delta t}{\Delta x^2}$.
Et donc d'un point de vu matriciel, le flux au niveau le plus fin avec un stencil $s=1$ s'écrit : 

\begin{align*}
    \doublehat{\Phi}_{k-1/2}
        &=
    \begin{bmatrix}
        -\frac{\lambda}{2}&
        (\frac{3}{2} \lambda - 1)&
        (1 - \frac{3}{2} \lambda)&
        \frac{\lambda}{2}&
    \end{bmatrix}
    \begin{bmatrix}
        +1/8 & 1 & -1/8 & 0\\
        -1/8 & 1 & +1/8 & 0\\
        0 & +1/8 & 1 & -1/8\\
        0 & -1/8 & 1 & +1/8\\
    \end{bmatrix}^{\dlbar}
    \begin{bmatrix}
        u^l_{k-2}\\
        u^l_{k-1}\\
        u^l_{k}\\
        u^l_{k+1}\\
    \end{bmatrix}
\end{align*}
\begin{align*}
    \doublehat{\Phi}_{k+1/2}
        &=
    \begin{bmatrix}
        -\frac{\lambda}{2}&
        (\frac{3}{2} \lambda - 1)&
        (1 - \frac{3}{2} \lambda)&
        \frac{\lambda}{2}&
    \end{bmatrix}
    \begin{bmatrix}
        +1/8 & 1 & -1/8 & 0\\
        -1/8 & 1 & +1/8 & 0\\
        0 & +1/8 & 1 & -1/8\\
        0 & -1/8 & 1 & +1/8\\
    \end{bmatrix}^{\dlbar}
    \begin{bmatrix}
        u^l_{k-1}\\
        u^l_{k}\\
        u^l_{k+1}\\
        u^l_{k+2}\\
    \end{bmatrix}.
\end{align*}

Attention le schéma est légèrement différent car il fait ici intervenir deux pas d'espace: $\Delta x$ le pas au niveau le plus fin
et $\Tilde {\Delta x} = 2^{\Delta l} \Delta x$ le pas du niveau courrant. Ainsi le schéma final est :
\begin{align}
    \doublehat{u}^{n+1}_k = \doublehat{u}^n_k + \frac{\lambda}{2^{\Delta l}} \Bigl( \doublehat{\Phi}^n_{k+1/2} - \doublehat{\Phi}^n_{k-1/2} \Bigr)
\end{align}
\newpage
\section{Équations équivalentes}
    Dans cette partie nous calculons les équations équivalentes du schéma numérique avec et sans multirésolution adaptative.
    L'équation équivalente d'un schéma est l'EDP dont la solution satisfait le schéma. Elle est calculée formellement par des développements de Taylor en temps et en espace.
    Comparer l'équation équivalente et l'équation cible fait alors naturellement aparaitre les termes d'érreur. 
    Dans les parties suivante beaucoup de calculs ont été réalisés grâce à librairie de calcul formeel Sympy\footnote{https://www.sympy.org}
    \begin{center}
\begin{tikzpicture}[node distance=1cm, every node/.style={font=\small}, >=latex]

% Nodes
\node (scheme) [draw, rectangle, fill=blue!10] {Schéma numérique};

\node (dl) [draw, rectangle, below=.5cm of scheme, fill=orange!10] {Développement de Taylor en espace et en temps};

\node (ck) [draw,dashed, rectangle, below=.5cm of dl, fill=orange!10] {Cauchy-Kowalevski};

\node (relation) [draw, dashed, rectangle, below=.5cm of ck, fill=orange!10] {Hypothèse : $\Delta t = f(\Delta x$)};

\node (eqeq) [draw, rectangle, below=.5cm of relation, fill=red!10] {Équation équivalente};

% Arrows
\draw[->, thick] (scheme) -- (dl);
\draw[->, thick] (dl) -- (ck);
\draw[->, thick] (ck) -- (relation);
\draw[->, thick] (relation) -- (eqeq);
% Titles
\node[above of=scheme, node distance=1.2cm, font=\bfseries] {Processus de d'obtenton d’une équation équivalente};

\end{tikzpicture}
\end{center}

    \subsection{Équation équivalente sans multirésolution}
    \begin{align}
    \frac{\partial u}{\partial t}  =&+ D \frac{\partial^{2}u}{\partial x^{2}} 
    + \Delta x^{2} \frac{D}{12}             \frac{\partial^{4}u}{\partial x^{4}} 
    -  \Delta t^{2} \frac{D^{3}}{6}          \frac{\partial^{6}u}{\partial x^{6}} 
    -  \Delta t^{3} \frac{D^{4}}{24}        \frac{\partial^{8}u}{\partial x^{8}}.
\end{align}
Le schéma de base est donc bien d'ordre deux en espace et en temps.
En supposant une relation de stabilité du type $\lambda = \frac{D\, \Delta t }{\Delta x^2}$:
\begin{align}
    \frac{\partial u}{\partial t}  =&\; D \frac{\partial^{2}u}{\partial x^{2}} 
    + \Delta x^{2} \frac{D}{12}             \frac{\partial^{4}u}{\partial x^{4}}
    - \Delta x^{4} \frac{D \lambda^2}{6}     \frac{\partial^{6}u}{\partial x^{6}}
    - \Delta x^{6} \frac{D \lambda^3}{24}    \frac{\partial^{8}u}{\partial x^{8}}.
\end{align}

    \subsection{Équation équivalente avec multirésolution}
    \begin{align}
    \frac{\partial u}{\partial t} =&\; D \frac{\partial^2 u}{\partial x^2} \\\notag
    &- \frac{\Delta t}{2} D^2\, \bigl( 2^{2\Delta l}- 1 \bigr)          \frac{\partial^4 u}{\partial x^4}
    - \Delta t^2\, \frac{D^3}{6}          \frac{\partial^6 u}{\partial x^6}
    - \Delta t^3\, \frac{D^4}{24}         \frac{\partial^8 u}{\partial x^8} \\\notag
    &+ 2^{2\Delta l}\, \frac{D\, \Delta x^2}{12}    \frac{\partial^4 u}{\partial x^4}
    - 2^{2\Delta l}\, \frac{D\, \Delta l\, \Delta x^2}{4} \frac{\partial^4 u}{\partial x^4}
\end{align}
Nous constatons que formellement le schéma est formellement d'ordre un.
Cela suggère donc que théoriquement, la multirésolution devrait faire perdre l'ordre de convergence temporelle de la méthode des lignes.
Cependant en pratique pour des raisons de stabilité on impose $\lambda= \frac{D \Delta t } {\Delta x^2}$.
Cela masque la perte en ordre\footnote{Cela à été vérifié expérimentalement} puisque cela mène à l'éqution équivalente:

\begin{align}
    \frac{\partial u}{\partial t}
    =&+ D \frac{\partial^{2}u}{\partial x^{2}}\\\notag
     &+ \Delta x^{2}\Bigl(\frac{2^{2 \Delta l} D  \lambda \frac{\partial^{4}u}{\partial x^{4}}}{2} 
     -  \frac{2^{2 \Delta l} D \Delta l \frac{\partial^{4}u}{\partial x^{4}}}{4} 
     -  \frac{D \lambda \frac{\partial^{4}u}{\partial x^{4}}}{2} 
     +  \frac{2^{2 \Delta l} D\frac{\partial^{4}u}{\partial x^{4}}}{12}\Bigr)\\\notag
     &- \Delta x^{6} \frac{D \lambda^{3} \frac{\partial^{8}u}{\partial x^{8}}}{24} 
     - \Delta x^{4} \frac{D \lambda^{2} \frac{\partial^{6}u}{\partial x^{6}}}{6} 
\end{align}

\newpage
\section{Mais d'où vient cette perte d'ordre ?}
    Un ordre de précision a été perdu en temps. Nous essayons a présent de comprendre quel mécanisme mène à cette baisse de performances.
    Ma démarche consiste à comparer les équations équivalentes avec et sans multirésolution, \textbf{avant} d'appliquer la procéduer de Cauchy-Kovaleskaya.
    \subsection{Équation équivalente sans multirésolution avant Cauchy-Kovaleskaya}
    L'équation modifiée sans multirésolution, avant procédure de Cauchy-Kovaleskaya est:
\begin{align}
    \frac{\partial u}{\partial t}  &= D \frac{\partial^{2} u}{\partial x^{2}} \\\notag
        &+ \frac{1}{2} \underbrace{\Bigl(D^{2}\frac{\partial^{4} u}{\partial x^{4}} - \frac{\partial^{2} u}{\partial t^{2}} \Bigr)}_{\substack{\text{Se compense par} \\ \text{la procédure de} \\ \text{Cauchy-Kovaleskaya}}} \Delta t
        + \frac{D}{12} \frac{\partial^{4} u}{\partial x^{4}}  \Delta x^{2}
        - \frac{1}{24} \frac{\partial^{4} u}{\partial t^{4}}  \Delta t^{3} 
        - \frac{1}{6}  \frac{\partial^{3} u}{\partial t^{3}}  \Delta t^{2}.
\end{align}
La méthode est bien d'ordre un, car à l'ordre un : $\frac{\partial u}{\partial t}  = D \frac{\partial^{2} u}{\partial x^{2}}$ et donc le terme $D^{2}\frac{\partial^{4} u}{\partial x^{4}} - \frac{\partial^{2} u}{\partial t^{2}}$
se compense au cours de la procédure de Cauchy-Kovaleskaya.
    \subsection{Équation équivalente avec multirésolution avant Cauchy-Kovaleskaya}
    L'équation modifiée avec multirésolution, avant procédure de Cauchy-Kovaleskaya est:
\begin{align}
    \frac{\partial u}{\partial  t} =&D \frac{\partial^{2} u}{\partial x^{2}}\\\notag
    &+ \frac{\Delta t}{2} \underbrace{\Bigl(2^{2 \Delta l} D^{2}           \frac{\partial^{4} u}{\partial x^{4}} -\frac{\partial^{2} u}{\partial t^{2}} \Bigr)}_{\text{Ne se compensent plus.}}
    -\frac{\Delta t^{3}}{24}                          \frac{\partial^{4} u}{\partial t^{4}} 
    - \frac{\Delta t^{2}}{6}                           \frac{\partial^{3} u}{\partial t^{3}}
    +\frac{\Delta x^{2}}{12} (1 - 3\Delta l)    2^{2 \Delta l} D \frac{\partial^{4} u}{\partial x^{4}}
\end{align}
Dans ce cas le terme en facteur du $\Delta t$ ne se s'annule plus. En effet le terme $D^{2}\frac{\partial^{4} u}{\partial x^{4}}$ est devenu au cours de la reconstruction
$2^{2 \Delta l} D^{2}\frac{\partial^{4} u}{\partial x^{4}}$. En conséquence, la méthode perds un ordre de convergence temporel.\par
Ce mécanisme s'explique de la manière suivante: dans l'équation équivalente, le terme $\frac{\partial^{2} u}{\partial t^{2}}$ apparaît indépendemment de la discrétisation spatiale
\footnote{Il emerge de la différence $u_k^{n+1} - u_k^{n}$ à $k$ fixé.}. La méthode des lignes initiale crée un terme \textit{sur mesure} pour le compenser en approximant le terme spatial
$D^{2}\frac{\partial^{4} u}{\partial x^{4}}$. Cepednant au cours du processus de compression-reconstruction, cette apprximation est entachée d'un facteur $2^{2 \Delta l}$.
En d'autres termes le terme spatial construit pour compenser un terme temporel a été modifié par la multi-résolution, alors que le terme temporel lui n'est pas affecté par la multirésolution.
Ainsi, les deux termes ne se compensent plus et l'ordre est perdu.

\section{Conclusion provisoire...}
    Cette étude mets en valeur qu'une méthode des lignes très simple, dans un contexte de multirésolution-adaptative
petu mener à des couplages des erreurs espace-temps polluant l'ordre initial de la méthode.
En particulier la reconstruction altère des termes spatiaux qui ne compensent plus certaines erreurs temporelles. 
Pour l'heure il conviendrait d'étudier ce phénomène sur d'autre méthodes, d'autres équations et surtout de l'étudier expérimentalement.

\newpage
\appendix
\section{Annexes}
\subsection{Principe général d'obtention des équations équivalentes.}
\paragraph{Première étape : développement de Taylor}
Dans un premier temps, on suppose l'existence d'une fonction assez régulière vérifiant le schéma.
Dans le schéma numérique les termes $u^{n+\delta_t}_{k+\delta_x}$ sont remplacés par leur pendant
continus: $u(x+\delta_x \Delta x, t + \delta_t \Delta t)$. Puis on réalise le dévelopmment de Taylor suivant : 
\begin{align}
    u(x+\delta_x \Delta x, t + \delta_t \Delta t) &= u(x+\delta_x \Delta x , t) 
        + \sum_{i} \frac{(\delta_x \Delta x )^j}{j!} \frac{\partial^j u}{\partial t^j }(x+\delta_x \Delta x)\\\notag
    &= u(x , t) 
        + \sum_i  \frac{(\delta_x \Delta x)^i}{i!} 
                \frac{\partial^i u}{\partial x^i}u(x,t)
        + \sum_j  \frac{(\delta_x \Delta x )^j}{j!} 
                \frac{\partial^j u}{\partial^j t}(x,t)
        + \sum_{i,j}  \frac{(\delta_x \Delta x )^{i}(\delta_t \Delta t )^{j}}  {i! \times j!}
                \frac{\partial^{j+i} u}     {\partial t^j \partial x^i}(x,t)
\end{align}
L'équation au dérivée partielles qui apparait est l'équation équivalente. libre à nous de la tronquer à l'ordre qui nous convient.
\paragraph{Deuxième étape : procédure de Cauchy-Kovaleskaya (optionnel)}
Afin d'enrichir l'analyse, et permettre le développement de schéma couplé espace-temps, l'étape précédente est généralement suivie d'une procédure de Cauchy-Kovaleskaya.
La procédure de Cauchy-Kovaleskaya consiste à utiliser la rélation entre les dérivées en espace et en temps données par l'équation cible et remplacer les dérivées en temps par des dérivées en espace dans l'équation équivalente. 
Cela consiste simple à partir de l'équation : $\dt{u} = D \dxx{u}$ et décrire de manière itérée : 
\begin{align}
    \dtt{u} &= D^2 \frac{\partial^4 u }{\partial x^4}\\\notag
    \dttt{u}&= D^3 \frac{\partial^6 u }{\partial x^6}\\\notag
    &\vdots\\\notag
    \frac{\partial^n u}{\partial x^n} &=D^n \frac{\partial^{2n} u}{\partial x^{2n}}
\end{align}
\paragraph{Ultime étape étape : relation entre le pas de temps et d'espace (optionnel)}
Lorsque l'on utilise un schéma en pratique, il est courrant de supposer une relation entre les pas d'espace et de temps, par exmeple une condition de stabilité du type 
$\Delta t \propto \frac{\Delta x^2}{D}$.
Il est donc très utile d'injecter cette relation dans l'équation équivalente pour comprendre le comportement du schéma en conditions réelles.

\printbibliography


\end{document}
